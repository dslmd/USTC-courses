\PaperTitle{用于玻色取样的自发参量下转换单光子源的优化(Optimizing spontaneous parametric down-conversion sources for boson sampling)} % Article title


\Authors{李明达\textsuperscript{1}} % Authors
\affiliation{
	\quad
	\textsuperscript{1}\textit{中国科学技术大学物理学院:学号PB18020616}
	\qquad
} % Author affiliation

\Abstract{
	\phantom{田田}本阅读报告首先调研了论文以外的一些知识点,从单光子源、HOM干涉实验、准相位匹配、自发参量下转换、非线性晶体到玻色取样,涵盖了读懂本论文所需要的所有量子光学与非线性光学知识。随后,笔者又采用一种相对于原论文更加有逻辑的方式来阐述该论文的脉络与思路,并给出这篇文章的结论——可以利用SPDC光子源实现“量子优越性”。最后,笔者从自己作为一个量子光学学习者的观点,来论述我对这篇文章的心得体会。
}


\Keywords{\phantom{田田}量子光学、非线性光学、单光子源、量子计算、玻色取样、自发参量下转换、非线性晶体、数值模拟} % 如不需要关键词可直接删去花括号中内容


